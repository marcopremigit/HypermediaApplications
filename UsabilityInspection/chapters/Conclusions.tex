\defChapterTarget{Conclusioni}
Il sito è ben strutturato e la presenza dei landmarks e shortcuts all'interno di 
ogni pagina in posizioni "vantaggiose", quali in cima e sulla barra laterale, rende molto
agevole e facile la navigazione. Tuttavia i link risultano essere in alcuni casi
eccessivamente ridodanti ed addirittura superflui.\\
Il risultato della nostra analisi mette in evidenza, sulla base delle
valutazione e dei commmenti dati, come le euristiche "Navigazione strutturale" e
"Disposizione del testo" siano dei punti di forza della pagina. Diversamente, l'euristica
"Sovraccarico di informazioni" relativa al contenuto ha ricevuto una valutazione bassa
ed alcuni commenti negativi, mettendo in evidenza uno dei problemi strutturali
principali del sito.\\
Per poter permettere una più agevole navigazione, di seguito riportiamo alcuni
consigli e miglioramenti per la struttura generale, e non, del sito
\href{https://www.visitmonterosa.com/}{\texttt{visitmonterosa.com}}:
\begin{itemize}
    \item \textbf{"Last news" in home page}: le informazioni riguardanti le
    strutture ed eventi intorno al comprensorio potrebbero essere un punto chiave di una
    ricerca da parte degli utenti del sito ed appassionati;
    \item \textbf{Suddivisione categorie e sottoelementi}: al momento la
    suddivisione di categorie ed i loro sottoelementi risulta alquanto
    dispersiva e poco intuitiva. Il suggerimento sarebbe quello di una
    migliore ricatalogazione per permettere una navigazione più immediata e di
    facile utilizzo;
    \item \textbf{Modifica link layout}: capita spesso e volentieri (durante la
    navigazione all'interno del sito) che alcuni link siano poco visibili o con
    uno sfondo non consono. Modificare il colore dei link o del background
    aiuterebbe in questo senso l'eventuale parziale perdita di informazioni.
\end{itemize}