\defChapterTarget{Introduzione al metodo}
    Per il nostro studio di usabilità abbiamo utilizzato alcune delle euristiche
    di ispezione proposte da MILE (Milano-Lugano evaluation method)
    \section{Euristiche specifiche}
        \subsection{Navigazione}
        \begin{itemize}
            \item \textbf{Navigazione}
            \begin{itemize}
                \item \textbf{Consistenza nelle interazioni}
                \item \textbf{Navigazione dei gruppi}
                \item \textbf{Navigazione strutturale}
                \item \textbf{Navigazione semantica}
                \item \textbf{Punti di riferimento}
            \end{itemize}
            \item \textbf{Contenuto}
            \begin{itemize}
                \item \textbf{Sovraccarico di informazioni}
            \end{itemize}
            \item \textbf{Layout}
            \begin{itemize}
                \item \textbf{Disposizione del testo}
                \item \textbf{Interazione dei placeholder}
                \item \textbf{Allocazione spaziale}
                \item \textbf{Consistenza della struttura delle pagine}
            \end{itemize}
        \end{itemize}
    \section{Metrica utilizzata} 
    Per ogni euristica abbiamo deciso di assegnare un voto intero tra 0 e 10,
    dove 0 indica che l'euristica non è soddisfatta, mentre 10 indica che
    l'euristica è pienamente soddisfatta.