\defChapterTarget{Metodo di Ispezione}
    \section{Panoramica}
    \begin{itemize}
        \item \textbf{Comprensione dell'euristica: }prima di procedere con
        l'analisi del sito abbiamo riconosciuto come centrale la scelta e la
        compresione delle euristiche. Certe euristiche sembrano infatti simili,
        ma in realtà presentano profonde differenze importanti da comprendere.
        \item \textbf{Definizione della metrica: }abbiamo definito un
        intervallo di valori per valutare le euristiche che ci permettesse di
        esprimere un giudizio preciso e in grado di valutare il sito nella
        maniera più corretta possibile.
        \item \textbf{Definizione degli obiettivi: }sulle base delle azioni che
        un generico utente sarebbe portato a fare durante la consultazione del
        sito abbiamo concordato degli obiettivi principali per l' analisi:
        \begin{itemize}
            \item Provare a prenotare per un pernottamente nella sezione "Dove Dormire"
            \item Muoversi nel sito per avere le ultime news e info utili
            \item Verificare la disponibilità degli impianti sciistici e comprare skipass
            \item Vedere quali sono i ristoranti
            \item Guardare quali sono i prossimi eventi a cui è possibile partecipare
        \end{itemize}
        \item \textbf{Analisi e report dei risultati individuali: }ogni
        esaminatore esplorando il sito individualmente e provando a rispettare
        gli obiettivi condivisi ha assegnato delle valutazioni alle singoli
        euristiche
        \item \textbf{Analisi e definizioni dei risultati finali: }abbiamo
        deciso di effettuare la media tra i valori proposti dai singoli
        esaminatori senza arrotondare per evidenziare in maniera più precisa la
        valutazione complessiva.
        \item \textbf{Reporting: }ogni esaminatore ha scritto degli appunti
        duranta l'analisi individuale e poi abbiamo proceduto a una stesura
        collettiva del documento.
    \end{itemize}
    \section{Euristiche}
    Per il nostro studio di usabilità abbiamo utilizzato alcune delle euristiche
    di ispezione proposte da MILE (Milano-Lugano evaluation method).
    \begin{itemize}
        \item \textbf{Navigazione}: la categoria "Navigazione" prende in
        considerazione le euristiche relative alla facilità di esplorazione del sito
        sito tramite i collegamenti e i link.
        \begin{itemize}
            \item \textbf{Consistenza nelle interazioni}: prende in esame le
            pagine dello stesso tipo e ne analizza la consistenza dal punto
            di vista dei link e della capacità interattive;
            \item \textbf{Navigazione dei gruppi}: considera la facilità con
            cui è possibile muoversi all' interno o attraverso elementi
            appartenenti allo stesso gruppo;
            \item \textbf{Navigazione strutturale}: riguarda la possibilità
            di potersi muovere tra componenti dello stesso topic;
            \item \textbf{Navigazione semantica}: indica se è possibile
            spostarsi agevolmente da un topic ad uno collegato
            semanticamente in entrambe le direzioni;
            \item \textbf{Landmarks}: riguardano la disponibilità di link
            tramite cui l' utente può raggiungere le pagine principali del
            sito;
        \end{itemize}
        \item \textbf{Contenuto}: le euristiche fornite all'interno della categoria
        "Contenuto" permettono di analizzare nel dettaglio le informazioni fornite
        dal sito all'utente e la facilità o meno con cui quest'ultimo riesce ad 
        elaborare le informazioni richieste.
        \begin{itemize}
            \item \textbf{Sovraccarico di informazioni}: quantità e qualità
            di informazioni che sono rese disponbili all' utente;
        \end{itemize}
        \item \textbf{Layout}: la categoria "Layout" riguarda gli aspetti
        stilistici e di design della pagina, usati allo scopo di mettere in
        evidenza all' utente le caratteristiche e le informazioni principali
        del sito web.
        \begin{itemize}
            \item \textbf{Disposizione del testo}: è un'euristica relativa
            alla leggibilità del testo all' interno delle diverse pagine;
            \item \textbf{Interazione dei placeholder}:  considera se il
            testo e gli elementi interattivi danno all' utente un' idea
            corretta del loro significato senza creare nessuna ambiguità;
            \item \textbf{Allocazione spaziale}: un'euristica relativa alla
            corretta disposizione di ciascun elemento, in base anche alla
            sua importanza, all' interno della pagina;
            \item \textbf{Consistenza della struttura delle pagine}: prende
            in analisi il Layout delle pagine che sono dello stesso tipo,
            considerando quindi se sono visivamente simili e se l'
            organizzazione dei vari elementi viene fatta nello stesso modo;
        \end{itemize}
    \end{itemize}
    \section{Metodo di valutazione}
    Per ogni euristica abbiamo deciso di assegnare un voto intero tra 0 e 10,
    dove 0 indica che l'euristica non è soddisfatta, mentre 10 indica che
    l'euristica è pienamente soddisfatta.