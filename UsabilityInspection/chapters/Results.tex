\defChapterTarget{Esecuzione e Risulati}
    Ogni esaminatore ha assegnato un voto
    per le singole euristiche prese in considerazione. I risultati presenti
    nelle tabelle presentate in questo documento sono ottenuti dalle medie dei
    valori proposti.
    I voti proposti sono giustificati dalle motivazioni condivise dagli esaminatori.
    \section{Navigazione}
        \begin{table}[H]
        \begin{tabular}{|l|l|l|l|l|}
        \hline \textbf{Euristica} & \textbf{Davide} & \textbf{Marco} & \textbf{Fabrizio} & \textbf{Voto totale} \\ \hline
        Consistenza nelle interazioni & 9.0 & 8.0 & 7.0 & 8.0 \\ \hline
        Navigazione dei gruppi & 8.0 & 8.0 & 7.0 & 7.7 \\ \hline
        Navigazione strutturale & 10.0 & 7.0 & 8.0 & 8.3 \\ \hline
        Navigazione semantica & 5.0 & 6.0 & 5.0 & 5.3 \\ \hline
        Punti di riferimento & 9.0 & 6.0 & 8.0 & 7.7 \\ \hline
        \end{tabular}
        \end{table}
        \subsection{Interaction Consistency}
        Le pagine dello stesso tipo hanno gli stessi links e la stessa capacità
        di interazione. Le pagine Monterosa SKi e Monterosa Discover sono
        infatti dello stesso tipo, in quanto risultano entrambe delle pagine
        principali, e sono strutturate nello stesso modo per quanto riguarda le
        tipologie di link e collegamenti.
        Non sempre questa euristica è rispettata in tutti gli elementi, ad
        esempio nella pagina "Monterosa Kids" non sono presenti i links riguardo
        le nuove proposte che sono invece presenti in Monterosa NatureCulture.
    

        \subsection{Group Navigation}
        La possibilità di spostarsi all’ interno di elementi appartenenti allo
        stesso gruppo è garantito dalla presenza dei subitems presenti nei
        landmark. La possibilità però di spostarsi da un elemento ad un altro
        dello stesso gruppo è solo in alcuni casi possibile, come ad esempio
        entrando nella pagina relativa agli appartamenti è sempre possibile
        sceglierne un'altra tipologia usando la voce tipo di struttura. In altri
        casi è necessario dover selezionare nuovamente il gruppo intero (sezione
        Monterosa ski). 
        \subsection{Structural Navigation}
        Il sito rende possibile l'esplorazione di ogni pagina partendo dai link
        presenti all’interno del corpo. Un esempio è  la pagina principale
        Discover Monterosa.
        Un altro esempio che possiamo considerare è la pagina Monterosa Active,
        dove i singoli elementi  (sci alpinismo, sci di fondo, ice climbing,
        ciasponel, sci, heliski) sono chiaramente evidenziati nella pagina. Sono
        inoltre raggiungibili anchè dal menu in testa al sito rendendo ancora
        più veloce raggiungere l’elemento di interesse. Tutto questo consente
        alla pagina Monterosa Active di non avere troppe informazioni al suo
        interno, dividendo il suo contenuto nelle altre pagine dedicate ai
        singoli elementi.
        Non sempre la suddivione dei subitem nei menu è intuitiva. Ad esempio la
        categoria "Monterosa Ski" presenta troppi elementi pochi chiari nella
        suddivione.
        \subsection{Semantic navigation}
        Non è facile muoversi da argomenti che hanno degli elementi in comune ad
        esempio per quanto riguarda le “lezioni di sci” ci si potrebbe aspettare
        oltre alla presenza del link nella pagina discovery anche la presenza di
        sub items all’ interno di monterosa Ski ma in quest'ultima non è
        presente. Un elemento positivo è dato dalla presenza dei link "Potrebbe
        interessarti".
        %semanticNavigation1
        Un elemento che risulta poco chiaro e in un certo senso anche
        “oppressivo” è l’onnipresenza della menu di prenotazione della vacanza
        sul lato destro delle pagine, presente anche in pagine che non
        dovrebbero averlo.
        %semanticNavigation2
        \subsection{Landmarks}
        I Landmarks sono molto utili per spostarsi all'interno del sito ed in
        particolar modo risultano essere importanti per tornare indietro nel
        caso si fosse entrati nella sezione sbagliata e sono in grado di mettere
        in evidenza i collegamenti che sono centrali all'interno del sito web.
        La presenza ulteriore di shortcuts all’ interno della pagina risulta
        essere un po’ eccessiva in quanto i collegamenti sono già individuabili
        all'interno dei “landmarks” presenti in alto nella pagina.
        %immagineDaFare
  
    \section{Contenuto}
    \begin{table}[H]
        \begin{tabular}{|l|l|l|l|l|}
        \hline \textbf{Euristica} & \textbf{Davide} & \textbf{Marco} & \textbf{Fabrizio} & \textbf{Voto totale} \\ \hline
        Sovraccarico di informazioni & 7.0 & 6.0 & 4.0 & 5.7 \\ \hline
        \end{tabular}
        \end{table}
        \subsection{Information Overload}
        Le diverse pagine del sito tendono sempre a una sovrabbondanza di
        elementi diversi e di menu. 
        %contenuto1
        Nella homepage ad esempio abbiamo davvero tanti elementi che tra l’altro
        sono un duplicato di quelli contenuti nel menu. Spesso ci sono molti
        subitems nei links del menu.
        %contenuto2
        Le informazioni relative al regolamento sebbene siano importanti sono
        descritte in modo eccessivamente prolisso.
        %contenuto3
        Tuttavia, oltre a queste caratteristiche, la pagina fornisce all'utente
        una visione globale di ciò che può essere trovato all'interno della
        pagina con sufficienti informazioni.


    \section{Layout}
        \begin{table}[H]
        \begin{tabular}{|l|l|l|l|l|}
        \hline \textbf{Euristica} & \textbf{Davide} & \textbf{Marco} & \textbf{Fabrizio} & \textbf{Voto totale} \\ \hline
        Disposizione del testo & 7.0 & 9.0 & 8.0 & 8.0 \\ \hline
        Interazione dei placeholder & 6.0 & 7.0 & 7.0 & 6.7 \\ \hline
        Allocazione spaziale & 7.0 & 5.0 & 6.0 & 6.0 \\ \hline
        Consistenza della struttura delle pagine & 7.0 & 8.0 & 8.0 & 7.7 \\ \hline
        \end{tabular}
        \end{table}
        \subsection{Text Layout}
        La dimensione e la tipologia di carattere utilizzato è leggibile, ma
        risulta essere leggermente piccolo come si può notare  dalle pagine
        “Piste di risalita per sci alpinismo” e “Iscrizione regolamento”.
        %textLayout1
        Ogni tanto il testo non è chiaramente leggibile, come in questa
        situazione:
        %textLayout2
        A causa della trasparenza il testo tende a confondersi con lo sfondo che
        presenza lo stesso colore. Il testo presenta generalmente una corretta gestione delle
        dimensioni che evidenzia una chiara distinzione delle gerarchie, come
        evidente da questo screenshot:
        %textLayout3
        Non sempre però la distinzione in gerarchie è correttamente rispettata, come ad esempio:
        %textLayout4
        \subsection{Interaction Placeholder}
        In certi casi la presenza di elementi interattivi è perfettamente
        evidente grazie all’utilizzo di colori diversi / elementi visivi
        caratterizzanti, come in questi casi:
        %interactionPlaceholder1
        In altri casi non è per nulla evidente.
        %interactionPlaceholder2
        In questo menu a tendina, ad esempio, non ho nessun elemento visivo in
        grado di farmi capire che se passo il mouse sopra “Monterosa NatureCulture” si
        aprirà un altro menu. La scoperta di questo menu avviene così in maniera casuale
        e non so mai quale link presenta un sottomenu o meno.
        %interactionPlaceholder3
        Discover Monterosa ed Experience Monterosa sono semanticamente simili e
        possono creare un po 'di ambiguità nella mente dell'utente riguardo a
        ciò che possono rappresentare.
        %interactionPlaceholder4
        Il link relativo alla parola “webcam” utilizzato all'interno della
        sezione Discover Monterosa non da un idea precisa del significato e
        della pagina che verrà ad aprirsi successivamente al click. La presenza
        inoltre di un elemento Ski all'interno di Experience Monterosa crea
        ambiguità in quanto ci si aspetterebbe di trovare le stesse informazioni
        relative a quelle contenute nella pagina Monterosa Ski.
        \subsection{Spatial Allocation}
        Certi elementi visivi tendono a essere dominanti su altri anche quando
        non dovrebbero esserlo. Qui per esempio sono nella pagina dedicata alle
        attività sul monterosa e visivamente il testo che spiega le attività è
        nascosto rispetto a tutto quello che lo circonda, quando invece dovrebbe
        essere l’elemento principale. Il menu laterale non è inoltre collegato
        semanticamente perchè la prenotazione dovrebbere essere in una pagina
        dedicata a quello.
        %spatialAllocation1
        La presenza della barra laterale a sinistra inoltre è un elemento che
        riduce la Unity poichè sposta tutto a sinistra senza avere un
        corrispettivo a destra. Per quanto riguarda la Gestalt invece c’è una
        chiara divisione tra le sezioni. Nei landmarks tutte le informazioni
        relative al Monte Rosa, quindi i link Monterosa Experience, Monterosa
        Discovery, Monterosa Ski, devono essere vicine l'una all'altra in quanto
        riguardano semanticamente lo stesso argomento. Le news e le info utili
        inoltre dovrebbero essere vicini essendo legate entrambe dal fine di
        dare informazioni. La Barra relativa al “Cerca nel Sito” pur essendo
        utile all'utente per la ricerca risulta essere scomoda da trovare in
        quanto risulta essere posizionata in basso a destra
        \subsection{Consistency of Page Structure}
        La pagina Experience Monterosa e Discover Monterosa non hanno lo stesso
        layout anche se sono appartenenti allo stesso argomento. Un altro
        esempio sono le pagine presenti all'interno  dei link “servizi” (
        entrando da Monterosa Discovery), la sottocategoria di “visit Monterosa”
        “ristoranti e bar” e  “shopping” hanno ad esempio layout differenti pur
        appartenendo allo stessa tipologia di pagina.
        %consistency1
        %consistency2


        