\usepackage{amsmath}
\usepackage{amssymb}
\usepackage{graphicx} % includegraphics 
\usepackage{todonotes} % todo notes
\usepackage{bm} % bold math
\usepackage{enumitem} % enumerate alignment
\usepackage{hyperref} % urls
\usepackage{pgffor} % foreach

% \begin{uml %
\usepackage{tikz}
\usepackage{aeguill}
\usepackage{float}
\usepackage{caption}

\definecolor{plantucolor0000}{RGB}{255,255,255}
\definecolor{plantucolor0001}{RGB}{0,0,0}
\definecolor{plantucolor0002}{RGB}{254,254,206}
\definecolor{plantucolor0003}{RGB}{168,0,54}
\definecolor{plantucolor0004}{RGB}{238,238,238}
% } %

% \begin{tabular style %
\usepackage[thinlines]{easytable}
\usepackage{makecell}
\renewcommand\theadfont{\bfseries}
\usepackage{spreadtab}
% } %

% \begin{custom operators %
\DeclareMathOperator*{\GO}{G}
\DeclareMathOperator*{\REQ}{R}
\DeclareMathOperator*{\DA}{D}
\DeclareMathOperator*{\AR}{AR}
\DeclareMathOperator*{\PR}{PR}
\DeclareMathOperator*{\SC}{S}

\newcommand{\goal}[1]{\langle\bm{\GO_{#1}}\rangle}
\newcommand{\requirement}[1]{\langle\bm{\REQ_{#1}}\rangle}
\newcommand{\assumption}[1]{\langle\bm{\DA_{#1}}\rangle}
\newcommand{\arequirement}[1]{\langle\bm{\AR_{#1}}\rangle}
\newcommand{\prequirement}[1]{\langle\bm{\PR_{#1}}\rangle}
\newcommand{\scenario}[1]{\langle\bm{\SC_{#1}}\rangle}

\newcommand{\R}{\mathbb{R}}
\newcommand{\N}{\mathbb{N}}
\newcommand{\Z}{\mathbb{Z}}
% } %

% \begin{custom targets %
\usepackage{xifthen}
\usepackage{xparse}
\NewDocumentCommand{\defTarget}{o m}{
    \IfNoValueTF{#1}{
        \hypertarget{#2}{} #2
    }{
        \hypertarget{#1}{} #2
    }
}
\NewDocumentCommand{\defSectionTarget}{o o m}{
    \IfNoValueTF{#1}{
        \IfNoValueTF{#2}
            {\hypertarget{sec#3}{} \section{#3} \label{sec#3}}
            {\hypertarget{sec#2}{} \section{#3} \label{sec#2}}
    }{
        \IfNoValueTF{#2}
            {\hypertarget{sec#3}{} #1{#3} \label{sec#3}}
            {\hypertarget{sec#2}{} #1{#3} \label{sec#2}}
    }
}
\NewDocumentCommand{\defChapterTarget}{o m}{
    \chapter{#2}
    \IfNoValueTF{#1}
        {\hypertarget{chap#2}{}}
        {\hypertarget{chap#1}{}}
}

\ExplSyntaxOn
\NewDocumentCommand{\xforeach}{s m +m}{
    \IfBooleanTF{#1}{
        \clist_map_inline:on{#2}{#3}
    }{
        \clist_map_inline:nn{#2}{#3}
    }
}
\ExplSyntaxOff

\newcommand{\urlink}[1]{\href{#1}{\texttt{#1}}}
% } %

% \begin{custom section ref %
\newcommand*{\fullref}[1]{
    \hyperref[{#1}]{\autoref*{#1} \nameref*{#1}}
}
% } %

% \begin{chapter style %
\usepackage[T1]{fontenc}
\usepackage{titlesec, color, blindtext}
\definecolor{gray75}{gray}{0.75}
\newcommand{\hsp}{\hspace{20pt}}
\titleformat{\chapter}
    [hang]{\Huge\bfseries}
    {\thechapter\hsp\textcolor{gray75}{|}\hsp}
    {0pt}
    {\Huge\bfseries}
    
\setcounter{secnumdepth}{4}
% } %

% \begin{custom label ref %
\newcounter{labelCounter}
 
\makeatletter
\newcommand{\labelText}[2]{
    \refstepcounter{labelCounter}
    \immediate\write\outputstream{\unexpanded{#1}}
}
\makeatother

\newcounter{referencesCounter}
\stepcounter{referencesCounter}
\newcommand{\defRef}[1]{
    \hyperlink{label::text::\arabic{referencesCounter}}{$\left[\arabic{referencesCounter}\right]$}
    \makebox[0pt]{
        \labelText{#1}{label::text::\arabic{referencesCounter}}
        \stepcounter{referencesCounter}
    }
}
% } %